\item \documentclass[12pt]{article}

  % Get right packages
  \usepackage{amsmath}
  \usepackage{amssymb}
  \usepackage{amsthm}
  \usepackage{float}
  \usepackage{fullpage} % Package to use full page
  \usepackage{graphicx}
  \usepackage{hyperref}
  \usepackage{parskip} % Package to tweak paragraph skipping

  % User commands
  \newtheorem{thm}{Theorem}
  \DeclareMathOperator*{\argmax}{argmax}

  % Title info
  \title{UMA Protocol: Additional examnination of voting rewards}
  \author{UMA Project}
  \date{\today}

\begin{document}

\maketitle
\clearpage
\newpage


\section{Overview}

  As UMA continues to evolve, we have continued to revisit and improve many of our original processes. One of the
  processes that we think could use an update is the voting reward procedure. This document briefly discusses the
  current reward structure, the weaknesses of this approach, and a few alternatives to replace the current voting
  reward structure.


\section{Voting game}

  To give some loose structure, we (briefly) describe a game that could be used to formalize some of the discussion
  that follows later in this document.

  Consider a single UMA voter who holds $u$ UMA tokens with the total supply given by $U$. This is a game that lasts $T$
  periods. In each period there is a valid vote that should be resolved with probability $p$, but the voter can create
  a frivolous vote at cost $c_{f}$. We could incorporate a price impact of frivolous votes as well, but we leave this
  as work for the future. We also ignore the bribery game (which we discuss in other documents) and assume that
  all voters reveal the state of the world accurately. We also allow for a cost of participation in a vote is $c_{p}$.
  The number of UMA tokens that participate in a given vote is given by $0 \leq \xi \leq 1$. The voters who
  participate are rewarded with $\gamma(u, U, \xi)$.

  The voters objective is to maximize the total value of the number tokens that they have, i.e. $\max q_{T} u_{T}$ where
  $q_{T}$ is the price of an UMA token.


\section{Current vote reward program}

  The current vote rewards system is a flat inflation rate per vote. This means that

  $$\gamma(u, U, \xi) = \frac{u}{\xi U} \pi U = u \frac{\pi}{\xi}$$

  It is easy to see that if $u \frac{\pi}{\xi} > c_{f} + c_{p}$ then there are incentives to call frivolous votes for
  the sake of retrieving the rewards. This positive incentive to call frivolous votes is problematic for UMA's current
  incentive structure because the cost of calling a frivolous vote, $c_{f}$ is relatively low which means that UMA
  holders are economically incentivized to call meaningless votes. In practice we haven't seen much of this yet. We
  think this is because people recognize that calling these votes would degrade the system and weaken UMA, but the
  incentive remains and a less forward-looking party might decide to try ``pulling this lever''.

  We think there are three main drawbacks to the frivolous votes:

  \begin{enumerate}
    \item \textit{Bandwidth}: If many votes are being called, it requires more bandwidth from both the UMA team and
          from external parties. One could argue that this increased bandwidth requirement raises the probability
          that a mistake is made, but, even if the probability of a mistake were constant, the elevated number of
          frivolous votes raises the probability that a mistake is made simply through more opportunities to make a
          mistake.
    \item \textit{Inflation}: The raised vote numbers cause the system to inflate inconsistently and may also add
          pressure to the token which weakens our security mechanisms.
    \item \textit{Speed}: If UMA holders are encouraged to call frivolous votes then the key insights of the
          Optimistic Oracle are weakened because nothing will be settled optimistically. The inability to settle
          contracts optimistically will slow down our response time which makes our product less useful.
  \end{enumerate}

  In the coming sections, we propose modifications to the voting reward system to address these issues.


\section{Raise fees}

  One way that we could maintain the current reward structure is to examine, ``how high would the fees need to be in
  order to discourage the frivolous votes''? For example, if we assume that the highest (non-UMA) token holder is 5\%
  then that particular holder would have about 5,000,000 UMA. This means that they receive $0.05\% \times 5,000,000$
  UMA per vote which, at current prices, would be worth about \$25,000. If we set the dispute bond above this then only
  holders who have more than 5,000,000 UMA would be incentivized to create frivolous disputes.

  This would be the easiest to implement because it only requires modifications to the parameterizations that govern
  the Optimistic Oracle. However, it also has a high cost of maintenance because as the price of UMA fluctuates, the
  dispute bond would need to also need to fluctuate to ensure that the frivolous voting was economically irrational.


\section{Escalation game}

  Another proposal would be to implement a type of escalation game on the vote procedure. For example, we could force
  the people who call votes to post a bond. Another person could post a second bond of higher value to dispute whether
  the vote needed to occur. This could then be disputed in similar fashion etc...

  This does lower the incentive to call frivolous votes, but is a relatively complex system to implement. Unless a
  terminal state were imposed, this could also be played endlessly and an individual with sufficiently deep pockets
  could always buy their way out of a mistake.

  A drawback of this approach is that it would require a large commitment from engineering and Dev-X since it would
  involve defining the escalation game, modifying how the optimistic oracle works, and putting that logic into a new
  set of contracts.


\section{Time-based rewards}

  A final way that we might address these frivolous votes is by decoupling rewards from the number of votes. Rather
  than pay rewards based on a per-vote basis, we could pay rewards on a per-unit-time basis. The fact that the total
  number of tokens rewarded per unit of time is fixed means that users have no reason to attempt to call frivolous
  votes because it doesn't have a meaningful effect on their rewards.

  An example of how this could be implemented is to starty by inflating the total number of UMA tokens by 0.25\% per
  month which would be equivalent to 5/votes per month under the current system. UMA holders begin each month by
  locking\footnote{This locking mechanism isn't necessary, but it provides at least 1 action per month that justifies
  rewards being paid for acknowledging that they are ``on-call'' for the current time period. An added bonus it that we
  could potentially remove the snapshotting requirements for the UMA token since the UMA would be locked in a separate
  contract that already knew how much each user deposited} the UMA that they would like to use for voting in a
  particular contract. They would then participate as usual in any votes that occur during that time period. Their share
  of the reward would depend on the number of UMA tokens locked along with the number of times that they voted with the
  majority.

  The drawback of this approach is that it would require a larger commitment from engineering since it would involve a
  redesign of the token and voting mechanisms.


\section{Conclusion}

  This document raises concerns about the way that voting rewards are currently earned and proposes three alternatives
  to the current system:

  \begin{enumerate}
    \item Raise fees
    \item Escalation game
    \item Time-based rewards
  \end{enumerate}

  Of these options, I think the most promising is to move to a time-based reward system. This would require certain
  changes to the UMA token but I think it most effectively addresses the underlying concerns in the most generic and
  future-proof way.

\end{document}
