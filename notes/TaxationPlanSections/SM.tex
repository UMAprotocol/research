%!TEX root = ../TaxationPlan.tex

In the previous two models discussed there was no sense in which future margin was unknown.
Obviously in practice, we have significant uncertainty about how margin will evolve and we believe
it is important to incorporate this uncertainty into the model. We do this by assuming that margin
will evolve according to a Markov process.

One additional difference that will show up in this section is a separation between the amount of
taxes collected, $T_t$, and the size of the buyback, $X_t$. In the deterministic model, there were
no unexpected fluctuations or growth and margin was either constant or always growing. These two
features made it unnecessary to store some of the taxes collected today to pay for future buybacks.

We will assume that margin follows a Markov process, that is, that tomorrow's margin only depends on
what margin was today.

$$M_{t+1} = f(M_t, \varepsilon_{t+1})$$

A Markov process like this can be quite general and can special case both of our previous sections.
The only difference is that now, margin will begin low and then will (potentially) grow to higher
levels with some uncertainty about the path it takes to get there.

Our objective will be to choose $\{(X_t, T_t)\}_{t=0}^{\infty}$ to optimize certain goals. In this
document, we will focus on minimizing the present discounted value of payouts to token holders and
keeping tax rates at a low and consistent level\footnote{We do this by using a quadratic punishment
function which punishes any deviation of the tax rate from 0 quadratically. The quadratic structure
ensures that large taxes are penalized more heavily than smaller taxes --- Under certain
assumptions, such a structure often produces a solution in which taxes tomorrow are the same as the
taxes today in expectation}.

\textbf{Two Part Solution}

We will approach this problem by breaking it into two parts:

\begin{enumerate}
  \item Solve for the minimum cost policy function $X^*(M^t)$
  \item Find the tax function $T^*(M^t)$ that meets our objective of low and non-volatile taxes such
        that we can fund any sequence of buybacks, $\{X(M^t)\}$, without debt
\end{enumerate}

\textit{Buyback Policy}

In the discussion that follows, we will focus on the case in which there are no negative buybacks.
There are a few interesting features associated with the unconstrained case, but we relegate their
discussion to Appendix \ref{app:nbb}. The constrained program can be written as

\begin{align*}
  \min_{\{X_t\}} \; & E \left[ \sum_{t} \left(\frac{1}{1 + r} \right) X_t \right] \\
  &\text{subject to} \\
  \quad & 2 PfC_t \leq E \left[ \sum_{s=0}^{\infty} \left(\frac{1}{1 + r}\right)^s X_{t+s} \right] \quad (\lambda_t) \\
  \quad & X_t \geq 0 (\mu_t)
\end{align*}

If we add the restriction that the policy is Markov, $X^*(M^t) = X^*(M_t)$, then we know

\begin{align*}
  2 PfC_t &\leq E \left[ \sum_{s=0}^{\infty} \left( \frac{1}{1+r} \right)^s X^*(M^s) \right] \\
  &\leq \sum_{s=0}^{\infty} \left( \frac{1}{1+r} \right)^s E \left[X^*(M_s) | M_t \right]
\end{align*}

Finally, if we assume that the margin process, $\{M_t\}$, follows a discrete Markov chain with $N$
states, then the objective function can be simplified to

\begin{align*}
  &E \left[ \sum_{t} \left(\frac{1}{1 + r} \right) X_i(t) \right] \\
  &\Rightarrow \pi_0 (I - \frac{1}{1+r}P)^{-1} \vec{X}
\end{align*}

where $\pi_0$ is a vector that denotes the initial distribution across the states of margin and
$\vec{X} \equiv \{X_1, \dots, X_i, \dots, X_N\}$ denotes a buyback for each of the margin states.
Our program can then be written as a linear program:

\begin{align*}
  \min_{\vec{X}} \; & \pi_0 (I - \frac{1}{1+r}P)^{-1} \vec{X}
  &\text{subject to} \\
  \quad & 2 PfC_i \leq (I - \frac{1}{1 + r}P)^{-1} X_i \quad (\forall i) \\
  \quad & X_i \geq 0 \quad (\forall i)
\end{align*}

Given the model parameters, we can solve this program with any standard linear program solver.

\textit{Tax Policy}

We can then formalize the second step with

\begin{align*}
  \min_{\{T_t\}} \; & E \left[ \sum_{t} \left( \frac{1}{1 + r} \right)^t T_t^2 \right] \\
  &\text{subject to} \\
  T_t + (1 + r) D_t &\geq D_{t+1} + X^*_t \quad (\mu_t) \\
  D_t &\geq 0
\end{align*}

where $D_t$ denotes the amount that is currently stored in the rainy day fund. We can write this
recursively as

\begin{align*}
  V(D_t, M_t) &= \max_{T_t} \; T_t^2 + \frac{1}{1 + r} E [V(D_{t+1}, M_{t+1})] \\
  &\text{Subject to} \\
  &D_{t+1} + X_{t} \leq (1 + r) D_t + T_t \\
  &D_t \geq 0
\end{align*}

Although most quadratic problems have near analytical solutions, we cannot exploit them in this case
due to no borrowing inequality constraint on the rainy day fund because it adds a non-linearity to
the problem. We can still compute a solution to this recursive program using a "brute-force" style
method called value function iteration.

The output of such a solution will be a rule, $T^*(D_t, M_t)$, which expresses the tax that should
be imposed as a function of the amount currently stored in the rainy day fund and the current margin
in the system. Additionally, this "policy function" will imply a law of motion for the size of the
rainy day fund

\textbf{Numerical Example}

In this subsection, we describe a single numerical example. This will help us in the following
subsection when we discuss the insights that come out of this model.

There are some issues that arise when we set too fine of a time-scale, so for now we focus on a
montly scale. We set most of the parameters to our ``standard'' assumptions:

\begin{itemize}
  \item $\chi = \frac{1}{2}$: Corrupted with half of the votes
  \item $\eta = 0$: Full participation
  \item $\gamma = \frac{1}{2}$: Half of the margin is vulnerable to being stolen
  \item $r = 0.0021$: Monthly interest rate which annualizes to about 2.5\%
\end{itemize}

The only remaining question is how to pick our Markov process for $M_t$. We use a discrete Markov
approximation of the Logistic Growth process with some added disaster risk\footnote{When we say
disaster risk here, we mean that there's an absorbing state with margin at 0 and there's some
probability that the process reaches that state with positive probability}. We describe how we
generate this approximation in Appendix \ref{app:dmc} and leave it to the interested reader to
investigate further. We plot many possible histories of this process below to help with the
visualization of the implied outcomes associated with this process.

\begin{center}
  \begin{figure}[H]
    \scalebox{0.65}{\includegraphics{./TaxationPlanImages/StochasticMarginGrowth.png}}
    \label{fig:sm_stochastic_margin_growth}
  \end{figure}
\end{center}

We first illustrate how the buybacks vary by the current level of margin. Note that this is very
similar to what we observed in the deterministic case --- At low levels of margin, we can support
a policy of zero buybacks because there is positive probability that the margin will grow to a point
where there will be relatively large buybacks. In the image below, we annualize the monthly buybacks
as a percent of margin

\begin{center}
  \begin{figure}[H]
    \scalebox{0.65}{\includegraphics{./TaxationPlanImages/StochasticBuybacks.png}}
    \label{fig:sm_stochastic_buybacks}
  \end{figure}
\end{center}

We now demonstrate how taxes vary by the status of the rainy day fund and the current margin using
a heatmap. Current margin increases along the x-axis, the rainy day fund increases along the y-axis,
and the color of the image at a point demonstrates how high taxes are at that point.

% TODO: Add tax figure

\textbf{Tax Insights}

In this subsection, we discuss some of the findings associated with our numerical examples:

\textit{If there are no rainy day funds, taxes will typically reflect full buyback amount}

This point follows almost immediately from the fact that we have chosen to disallow any borrowing by
the DVM. If the rainy day fund, $D_t$, is currently at 0 then

\begin{align*}
  D_{t+1} + X_{t} \leq (1 + r)*D_t + T_t \\
  D_{t+1} + X_{t} \leq T_t
\end{align*}

The question is whether we will raise extra taxes to not experience the same state tomorrow. It
turns out for most of the possible states of margin, the optimal response is to set $T_t = X_t$ with
the result being $D_{t+1} = 0$. This means that if the rainy day fund ever arrives at 0 when there
are positive buybacks, that it might be hard to accumulate a rainy day fund

\textit{}
