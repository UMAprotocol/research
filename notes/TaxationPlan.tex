\documentclass[12pt]{article}

  % Get right packages
  \usepackage{amsmath}
  \usepackage{amssymb}
  \usepackage{amsthm}
  \usepackage{fullpage} % Package to use full page
  \usepackage{hyperref}
  \usepackage{parskip} % Package to tweak paragraph skipping

  % User commands
  \newtheorem{thm}{Theorem}
  \DeclareMathOperator*{\argmax}{argmax}

  % Title info
  \title{Taxation Notes}
  \author{UMA Project}
  \date{\today}

\begin{document}

\maketitle


\section{Summary}

  This document contains the thought process behind the thought flow on how the UMA taxation
  policy should work. Each section contains a slightly different version of a model that can be used
  to understand the tradeoffs induced by the tax

  The sections include:

  \begin{itemize}
    \item Section \ref{sec:dss} asks the question, ``In a deterministic world in which the system
          margin has reached a steady state, what is the lowest tax rate we can implement that
          ensure incorruptibility?''
    \item Section \ref{sec:dg} asks the question, ``If we knew exactly how margin would grow
          could charge in each period and maintain the incorruptibility gurantee
          starting today and going into the future, what is the minimum amount of taxes that we
          could impose at each period?''
    \item Section \ref{sec:sss} analyzes a model in which margin follows a stationary stochastic
          process and thinks about the right way to smooth taxes while making sure that payments
          can be made to ensure system is incorruptible.''
  \end{itemize}

  The main constraint in the choice of tax collected/buybacks conducted in each period is that the
  profit from corrupting ($PfC$) the oracle is less than the cost of corruping ($CoC$) the oracle.
  If corrupting the oracle requires a majority vote\footnote{This assumes the voting process simply
  ellicits a yes or no question, but similar analysis can be done for a more generic vote procedure}
  then this boils down to simply requiring that the cost of holding half of the tokens is less than
  the $PfC$ which will be a combination of the margin in system and any excess funds held in a
  ``rainy day fund.''

  Let $\eta$ represents the percentage of tokens that do not vote in a given period, etc.

  We will use following notation in these notes

  \begin{itemize}
    \item $B_t$ denotes the amount of tokens bought back prior to vote
    \item $D_t$ denotes the amount of funds held in the ``rainy day fund''
    \item $M_t$ denotes total margin in system
    \item $\tau_t$ denotes the tax rate on the system
    \item $p_t$ denotes the price of a single vote token
    \item $S_t$ denotes the number of vote tokens
    \item $T_t \equiv \tau_t M_t$ denotes total tax levied on system
    \item $X_t \equiv B_t p_t$ denotes the amount of expenditure on buybacks
  \end{itemize}

  Additionally the following are parameters

  \begin{itemize}
    \item $\eta$ is percent of people who do not vote
    \item $\pi$ is the inflation rate used to reward voters
  \end{itemize}

  \subsection{Price such that $PfC < CoC$}

    In this section we determine what price is needed in a generic period to ensure that
    $PfC < CoC$.

    \begin{align*}
      PfC_t \leq CoC_t \rightarrow PfC_t \leq \frac{1 - \eta}{2} p_t S_t \\
      \rightarrow p_t \geq \frac{2}{1 - \eta} \frac{PfC_t}{S_t}
    \end{align*}


\section{Deterministic Steady State} \label{sec:dss}

  In this version of the model, we will consider the system margin being constant over time.

  Let the margin be given by $M_t = \bar{M}$, $PfC_t = \bar{PfC}$, and $X_t = \bar{X}$

  Recall that the price that ensures $PfC < CoC$ is denoted by
  $p_t \geq \frac{2}{1 - \eta} \frac{PfC_t}{S_t}$.

  It's important to note that not all of the variables will be constant in the SS since there is
  potentially non-zero inflation in the number of tokens. The number of tokens follows

  \begin{align*}
    S_{t+1} &= (1 + \pi) (S_t - B_t)
  \end{align*}

  However, $p_t S_t$ will be a constant since

  \begin{align*}
    p_t S_t &= \frac{2}{1 - \eta} \frac{PfC_t}{S_t} S_t \\
    &= \frac{2}{1 - \eta} PfC_t
  \end{align*}

  and since $PfC_t = \bar{PfC}$ then

  $$p_t S_t = \bar{pS} = \frac{2}{1 - \eta} \bar{PfC}$$

  Assume that we'd like to achieve a period-by-period return of $r$.

  \begin{align*}
    (1 + r) &= \frac{p_{t+1} (S_{t+1} + B_{t+1})}{p_t S_t} \\
    (1 + r) &= \frac{p_{t+1} S_{t+1} + X_{t+1}}{p_t S_t} \\
    (1 + r) &= \frac{\frac{2}{1 - \eta} \frac{PfC_{t+1}}{S_{t+1}} S_{t+1} + X_{t+1}}{\frac{2}{1 - \eta} \frac{PfC_{t}}{S_t} S_t} \\
    (1 + r) PfC_{t} &= PfC_{t+1} + \frac{1 - \eta}{2} X_{t+1} \\
  \end{align*}

  In the SS this means that

  \begin{align*}
    (1 + r) \bar{PfC} &= \bar{PfC} + \frac{1 - \eta}{2} \bar{X} \\
    &\rightarrow \bar{X} = \frac{2 \bar{PfC}}{1 - \eta} r
  \end{align*}

  If we assume that $\bar{PfC} = \frac{1}{2} \bar{M}$ and that there is full participation then this
  reduces to $\bar{X} = r \bar{M}$ which means that the tax rate is given by,

  $$\bar{\tau} \equiv \frac{\bar{X}}{\bar{M}} = r$$


\section{Deterministic Growth} \label{sec:dg}

  We now consider how to implement taxes in a world where the system margin grows over time
  according to a deterministic process. In particular, we assume that margin follows

  $$M_{t+1} = M_{t} + g M_{t} \left(1 + \frac{M_{t}}{\bar{M}} \right)$$

  This process, known as logistic growth, generates ``S-shaped'' growth. We can see the implications
  that this process has for the system margin in Figure XYZ.

  We assume that we'd like to charge the minimum amount of taxes while maintaining the system's
  incorruptibility. This produces the following program:

  \begin{align*}
    \min_{T_t} &E \left[ \sum_{t=0} \left(\frac{1}{1 + r} \right)^t T_t \right] \\
    &\text{subject to} \\
    PfC_t &\leq \frac{1}{2} P_t S_t = \frac{1}{2} E \left[ \sum_{s=0} \left(\frac{1}{1 + r}\right)^s  X_{t + s} \right] \\
    X_{t} &= T_t \\
    0 &\leq T_t \\
    T_t &\leq \bar{\tau} \bar{M} \\
    M_{t+1} &= M_{t} + g M_{t} \left(1 + \frac{M_t}{\bar{M}} \right)
  \end{align*}

  The way that we solve this is that we start at $s$ such that $M_s \approx \bar{M}$. At this
  point, we have arrived in the steady state and the tax rate implemented will be
  $T_t = \bar{\tau} M_t$. We can then step back by one period to $t = s - 1$ and compute what the
  required $X_t$ in that period would be. We can proceed to step this back until we reach our
  initial condition of $M_0$ which traces out a path of tax collections. This process generates
  a sequence of taxes that look like Figure XYZ.

  The interesting observation that corresponds to this process is that the tax rates can start
  relatively low and stay low for a prolonger period of time simply because people know that there
  will be growth in the system --- This future promise of increased buybacks in the future is
  enough to secure the system while it is small. However, this depends drastically on the fact that
  margin grows deterministically and, in practice, the future growth may not be as certain as
  predicted in this model (but we only need that the marginal token holder believes in the growth).


\section{Stochastic Steady State} \label{sec:sss}

  We


\section{Stochastic Growth} \label{sec:sg}


\end{document}
