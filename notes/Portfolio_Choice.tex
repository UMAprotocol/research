\documentclass[12pt]{article}

  % Get right packages
  \usepackage{amsmath}
  \usepackage{amssymb}
  \usepackage{amsthm}
  \usepackage{fullpage} % Package to use full page
  \usepackage{hyperref}
  \usepackage{parskip} % Package to tweak paragraph skipping

  % User commands
  \newtheorem{thm}{Theorem}
  \DeclareMathOperator*{\argmax}{argmax}

  % Title info
  \title{Portfolio Choice Problem}
  \author{}
  \date{10/22/2018}

\begin{document}

\maketitle


\section{General Model}

\subsection{Individuals}

  Individuals purchase and sell vote tokens, $c$, and risk-free bonds with return $r$, $b$. They are
  risk-averse and have preferences over their wealth levels. They always attempt to vote truthfully,
  but may not do so for two reasons (1) forgetting to vote (non-participation) and (2) make a
  mistake when they try to vote (trembling hand). The probability with which they forget to vote is
  constant and given by probability $\eta$. The probability with which they make a mistake is
  distributed according to a beta distribution with parameters $\alpha, \beta$.

  An individual entering a period with $b_t$ bonds and $c_t$ coins faces the following problem:

  \begin{align}
    V(b_t, c_t, x_t) &= \max_{b_{t+1}, c_{t+1}} u(w_t) + \beta E_x \left[ V(b_{t+1}, c_{t+1}, x) \right] \\
    w_t &= (1 + r) b_t + c_t (1 + \gamma(x_t, X)) (d_t + p_t) - b_{t+1} - p_{t} c_{t+1}
  \end{align}


\subsection{Malevolent}

  The malevolent can also purchase and sell vote tokens. However, they may attempt to vote
  dishonestly if it is in their interest. The Malevolent is risk-neutral and attempts to maximize
  wealth.

  {\color{red} TODO: What happens if we just force the malevolent to always like? Does this prevent
  certain attacks, for example, buying early in life of asset and not corrupting until later? I
  think this would still be unprofitable because if they couldn't purchase and corrupt in that
  period then they'd be better off selling the coins than corrupting system}

\subsection{Oracle}

  The Oracle issues totaly supply 1 of the vote token. The Oracle wants to ensure that the true
  state is reported correctly. This is difficult because the Oracle has no way to observe what the
  state is, so it uses the votes of vote token holders to announce the state; if more than $\xi$
  percent of the vote tokens are used to cast a vote for a state then the Oracle announces that
  state. It rewards those who vote according to the payoff function $\gamma(x, \hat{S})$ which maps an
  individual's vote, $x$, and the predicted state, $\hat{S}$ to a new value between (0, 2).


\subsection{Equilibrium}

  An equilibrium is:

  \begin{enumerate}
    \item Value function $V : B \times C \times X \rightarrow \mathcal{R}$ for individuals
    \item Policy functions $b^*, c^* : B \times C \times X \rightarrow \mathcal{R}^{+}$
    \item Value function $\tilde{V} : B \times C \rightarrow \mathcal{R}$ for malevolent
    \item Policy functions $\tilde{b}^*, \tilde{c}^* : B \times C \rightarrow \mathcal{R}^{+}$ and $\tilde{x}^* : B \times C \rightarrow X$
    \item Oracle dividend function $d : M \rightarrow \mathcal{R}^{+}$
    \item Oracle redistribution function $\gamma : \{0, 1\} \times X \rightarrow [0, 2]$
    \item Oracle tax function $\tau : M \rightarrow [0, 1]$
    \item Prices $p$
  \end{enumerate}

  such that

  \begin{enumerate}
    \item Policy functions are optimal given value functions for both individuals and malevolent
    \item Prices clear the coins market -- $\sum_i c^*(b, c, x) = 1$
    \item Oracle redistribution keeps number of outstanding coins constant, $\pi \gamma(0, X) + (1 - \pi) \gamma(1, X) = 1$
    \item Malevolent finds it optimal to not attack
  \end{enumerate}

2. For small voters (their vote will not be pivotal with any significant probability) I agree there
is no need to solve a profit maximization problem to determine their vote. They can just vote what
they think is true, but maybe make mistakes (a parameter in the model will control the probability
of truthfully making mistakes). If the mistake probability is independent across voters and
symmetric (false positives and false negatives) then this all washes out in a law of large numbers
sense, but the mistakes may be correlated (some votes may actually be difficult to figure out what
the truth is). I would try to write the model generally, and then focus on a special case when doing
analysis in the early days, instead of starting with the special case directly. As is the general
theme right now, it seems good to have parameters that exogenously determine interesting things,
like probability of voter participation, which can then later be made endogenous. For large voters
(people who own lots of tokens), it may not be a good idea to make voting mechanical, and you may
want to consider profit maximization. This depends on what the market will ultimately look
like...lots of small token holders or a few big ones. Different market should be modeled
differently; I don't know what Hart envisions on this front.

3. Definitely like the malevolent purchasing tokens and acting strategically.

4. I agree no need to do mechanism design for now. Focus on a small set of tools/decision rules the
oracle can use for now.

5. I am not sure how difficult it is to endogenize the price of the tokens in the new or old model
under the assumption of risk neutrality of token owners and fixed supply of tokens (that are
perfectly divisible). Maybe you would need an outside risky rate of return for no arbitrage, but I'm
not sure that is needed, probably just a discount rate. I would assume perfect foresight for macro
variables (growth of the size of the margin in the system) and then you would need a voting rule for
each token holder and a probability of payoff for each event (non-vote, vote 1  and vote 1 wins,
vote 1 and vote 0 wins, vote 0 and vote 1 wins, vote 0 and vote 0 wins). Maybe the event space can
be collapsed to vote the winning vote or vote the losing vote, not sure. I'm not saying it is easy,
but I would like to see the equations that would determine the endogenous token price, even if it is
hard to solve that version of the model.

\end{document}
