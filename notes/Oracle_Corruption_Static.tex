\documentclass[12pt]{article}

  % Get right packages
  \usepackage{amsmath}
  \usepackage{amssymb}
  \usepackage{fullpage} % Package to use full page
  \usepackage{hyperref}
  \usepackage{parskip} % Package to tweak paragraph skipping
  \DeclareMathOperator*{\argmax}{argmax}


  % Title info
  \title{Oracle-Agent Problem}
  \author{}
  \date{10/22/2018}

\begin{document}

\maketitle


\section{Introduction}

  The UMA (Universal Markets Access) protocol is meant to provide universal access to financial
  markets so that everyone might have the opportunity to benefit from sharing risks with others. The
  way that we will achieve this is by building a platform for trading this risk and an
  \textit{oracle} that will dictate what market prices are for various forms of risk... In order to
  support having a non-trivial amount of risk on the UMA protocol, we need to ensure that the oracle
  has the ability to produce accurate price information, and, in particular, it must be resistant to
  bribery attacks where an individual attempts to move prices in their favor through corrupt means.

  This document describes a model which we use to influence decisions about how to ensure that the
  cost of corruption (CoC) is higher than the profit from corruption (PfC). The system is
  economically stable as long as the $\text{CoC} > \text{PfC}$. All things equal, it is best if the
  system generates a large spread between CoC and PfC to ensure that attacking the system requires
  as high a cost as reasonable, however, raising this spread will likely come at the cost of the
  efficiency of the system so it is important to think about the externalties imposed by the oracle
  on the trading protocol.

  We begin by describing a static environment that allows us to think about the tradeoffs faced when
  attempting to raise the CoC-PfC spread and ensuring that the system is difficult to corrupt.


\section{Static Environment}

  In the static world, the trading protocol generates $T = 2 \tau N$ revenue and has $mN$ margin
  that is exposed to being seized if the Oracle reports incorrect information\footnote{For the
  assumptions that generate these outcomes, see Appendix A}.

  There is a random state of the world $S \in \{0, 1\}$. There are three types of agents:

  \begin{enumerate}
    \item The oracle: Responsible for reporting $\hat{S}$. The oracle wants to choose $\hat{S} = S$.
    \item The malevolent: Can observe $S$ and receives $mN$ if the oracle reports $\hat{S} = 1 - S$.
    \item The individuals: Can observe $S$ and are incentivized by the oracle and the malevolent
    into voting $x = \{0, 1\}$.
  \end{enumerate}

  Agents choose whether to truthfully report the state. A truthful report entails $x = S$ and
  untruthful report is $x = 1 - S$. In order to participate in the vote, agents must purchase the
  right to vote at price $p$. In order to entice truthful reporting, the oracle is allowed to make
  payments to the agents, $\varepsilon(x, X)$ where $x$ is the agent's vote and $X$ number of people
  who voted $x_i = 0$. This payment is in terms of \textit{rights to vote for tomorrow} which are
  valued at $p' = \begin{cases} p \quad \text{if } \hat{S} = S \\ 0 \quad \text{else}\end{cases}$.
  The malevolent is also allowed to make payments to the agents, $\tilde{\varepsilon}(x, X, S)$, but
  can condition on the agent's vote, the number of votes for 0, AND the state.

  Agents report to maximize:

  $$V(x) = \max_{x \in \{0, 1\}} p' \varepsilon(x, X) + \tilde{\varepsilon}(x, X, S)$$

  The malevolent would like to corrupt the system as cheaply as possible. The cost at which the
  malevolent can corrupt the system is given by

  \begin{align*}
    C^M(S, \varepsilon(x, X)) &= \min_{\tilde{\varepsilon}(x, X, S)} \int_i \varepsilon(x_i, X, S) \\
    &\text{subject to} \\
    &V(1 - S) > V(S) \quad \text{(Lie Compatiblity)}
  \end{align*}

  The oracle would like to report the truth at the minimum cost. It's problem is given by

  \begin{align*}
    V^O &= \min_{\varepsilon(x, X)} \int_i \varepsilon(x_i, X) di \\
    &\text{subject to } \\
    &p' \int_i \varepsilon(x_i, X) di \leq T + \int_{i \in \text{Voters}} p di \quad \text{(Budget Constraint)} \\
    &V(S) \geq V(1 - S) \quad \text{(Incentive Compatible)} \\
    &C^M(S) \geq mN \quad \text{Malevolent Incompatible}
  \end{align*}

  The Malevolent Incompatible constraint is what ensures that the oracle chooses a payment scheme
  that cannot be corrupted because it ensures that $\text{CoC} > \text{PfC}$.


\section{Game Plan}

  The game plan is to solve for a discretized version of $\varepsilon(x, X)$ and determine the the
  corresponding $C^M(S, \varepsilon(x, X))$. With these in hand, we can likely pick a function that
  approximates $\varepsilon(x, X)$ ``well enough'' which mostly means that it doesn't require too
  much taxation on the trading system and that it satisfies the constraints of the oracle's problem.

  Once we have a solution to the static problem, we will move onto the dynamic problem. This will
  allow for $p'$ to be an endogenous object. The dynamic problem will also give the oracle more
  tools, such as positive reputation, to encourage truthful reporting. The malevolent agent does not
  have access to dynamic incentives because upon success the system shuts down. This should mean
  that the static results are the most expensive solution to the truthful oracle problem...

  
\section{Appendix A}

  To generate the assumption of $T = 2 \tau N$ and an amount of seizable margin of $mN$ we assume:

  \begin{enumerate}
    \item There are $N$ contracts held in the trading protocol.
    \item These contracts are symmetric in the sense that each counterparty holds $m$ margin.
    \item The malevolent is a counterparty on each of the outstanding contracts and if the wrong
    state is reported, the malevolent can extract each of their counterparty's margin.
    \item The counterparty of each contract is responsible for paying a tax, $\tau$, in order to
    be provided access to the oracle's information.
  \end{enumerate}

  Note that in some ways, this is a worst case analysis. The malevolent agent being a part of each
  contract is what drives the PfC to $mN$. In practice, the PfC would typically be lower than this.


\end{document}
