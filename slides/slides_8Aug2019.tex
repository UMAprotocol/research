\documentclass[10pt]{beamer}

  % Math Packages
  \usepackage{amsmath}
  \usepackage{amsthm}
  \usepackage{mathtools}

  % Graphics Packages
  \usepackage{graphicx}
  \usepackage{pgf}
  \usepackage[export]{adjustbox}

  % Colors
  \definecolor{umared}{RGB}{255,74,74}
  \definecolor{mygray}{RGB}{66,66,66}

  % Theme Settings
  \usetheme{metropolis}

  \setbeamercolor{normal text}{fg=mygray}
  \setbeamercolor{alerted text}{fg=umared}
  \setbeamercolor{title text}{fg=mygray}
  \setbeamercolor{title separator}{fg=umared}
  \setbeamercolor{progress bar}{fg=umared}
  \setbeamercolor{frametitle}{bg=umared}

\title{Oracle Taxation Theory}
\logo{
  \makebox[0.13\paperwidth]{\includegraphics[width=1.5cm,keepaspectratio]{Uma_Logo.png} \hfill}
}
\date[]{\today}

\begin{document}

% Title Slide
\begin{frame}
  \titlepage
\end{frame}


% --------------------------------------- %
% Goal
% --------------------------------------- %
\section{Goal Refresher}

  \begin{frame} \frametitle{$PfC < CoC$}

    The purpose of the taxation ($T_t$) and buyback policies ($X_t$) is to ensure that the price of
    the token prevents 51\% attacks. Formally, we want

    $$\underbrace{\gamma M_t}_{\text{Profit from Corruption}} < \underbrace{\frac{1}{2} p_t S_t}_{\text{Cost of Corruption}}$$

    This can be controlled by system because, if priced correctly, the token market cap should be

    $$p_t S_t = E \left[ \sum_{s=0}^{\infty} \left(\frac{1}{1 + r}\right)^s X_{t+s} \right]$$

  \end{frame}

% --------------------------------------- %
% Overview of Previous Material
% --------------------------------------- %
\section{Last Time}

  \begin{frame} \frametitle{Tax rates in deterministic world}

    We learned that, in a deterministic world, that tax rates

    \begin{itemize}
      \item Must eventually be roughly the same as the interest that we would like token holders to
            earn
      \item If people believe that the system margin will grow, then we can support relatively low
            taxes for a prolonged period of time
    \end{itemize}

  \end{frame}

% --------------------------------------- %
% What are we learning about today
% --------------------------------------- %
\section{Stochastic Model}

  \begin{frame} \frametitle{Why do we need stochastic?}

    In the deterministic model, there are no unexpected fluctuations or growth, and thus no risk

    In a model where margin grows without risk, there's no need to have a rainy day fund or to
    provide any other form of self-insurance

    In reality, we will need to ensure that there are funds to intevene and influence prices to
    ensure that $PfC < CoC$ --- To think about the right way to intervene, we need a stochastic
    framework

  \end{frame}

  \begin{frame} \frametitle{Framework}

    Margin, $M_t$ now follows a stochastic Markov process rather than a deterministic process

    $$M_{t+1} = f(M_t, \varepsilon_{t+1})$$

    Our goal will be to choose the ``right'' sequence of taxes to impose, $\{T_t\}$, and buybacks to
    make, $\{X_t\}$

  \end{frame}

  \begin{frame} \frametitle{What does ``right'' mean?}

    We have several, potentially competing goals to achieve with $\{T_t\}$ and $\{X_t\}$

    \vspace{0.25cm}

    \begin{itemize}
      \item First of all, need to ensure $PfC < CoC$
      \item Minimize the tax costs and volatility
      \item Minimize the cost of buyback policy by minimizing $\{X_t\}$ (do we also care about
            volatility here too?)
    \end{itemize}

  \end{frame}

  \begin{frame} \frametitle{Two steps to find policies}

    \begin{enumerate}
      \item Find the buyback policy, $X^*(M_t)$, which solves
        \begin{align*}
          \min_{\{X_t\}} \quad &E \left[ \sum_{s=0}^{\infty} \left(\frac{1}{1 + r}\right)^s X_{t+s} \right] \\
          &\text{Subject to } \\
          &PfC < CoC
        \end{align*}
      \item Once we have a buyback policy, we want to determine how to finance the buybacks
        \begin{align*}
          \min_{\{T_t\}} \quad &E \left[ \sum_{s=0}^{\infty} \left(\frac{1}{1 + r}\right)^s \frac{1}{2} T_t^2 \right] \\
          &\text{Subject to } \\
          &T_t + (1 + r) D_t \geq D_{t+1} + X_t^*(M_t)
        \end{align*}
    \end{enumerate}

  \end{frame}

  \begin{frame} \frametitle{Determine the Buybacks}

    In this context, a \textit{Markov Perfect Strategy} is a buyback policy,
    $X^*(M_t) : M \rightarrow \mathcal{R}$ which is time-invariant.

    \vspace{0.25cm}

    Effectively, the buyback policy must only be a function of current margin, $M_t$.

    \vspace{0.25cm}

    \textbf{Proposition}: There exists a MP strategy $X^*(M_t)$ such that there is no function
    $X^*(M^t)$ which achieves a lower cost of buybacks

  \end{frame}

  \begin{frame} \frametitle{Determine the Buybacks}

    Buyback function will satisfy

    $$2 \gamma M_t = \sum_{s=0}^{\infty} \left( \frac{1}{1+r} \right)^s E \left[X^*(M_{t+1}) | M_t \right]$$

    If $M_t$ follows a discrete Markov chain, can write $X^*(M)$ as

    $$X = 2 \gamma \left(I - \frac{1}{1+r} P \right) \begin{bmatrix} M_1 \\ M_2 \\ \vdots \\ M_N \end{bmatrix}$$

  \end{frame}

  \begin{frame} \frametitle{Two cases}

    \textbf{Case 1: Persistent Margin}

    In the case of relatively persistent margin, the buyback levels are relatively inelastic and do
    not move much relative to margin --- In one numerical example, for a \$100,000 change in margin
    the buybacks only change by about \$5,000.

    \vspace{0.5cm}

    \textbf{Case 2: Transitory Margin}

    In the case of transitory (not persistent) margin, the buyback levels respond much more
    strongly --- In one numerical example, for a \$100,000 change in margin the buybacks change by
    about \$75,000.

  \end{frame}

\end{document}
